\documentclass[11pt]{article}
\usepackage[margin=1.2in]{geometry}
% \usepackage{showframe}
\usepackage{caption}
\usepackage{float}
\usepackage{enumerate}
\usepackage{array}
\usepackage{graphicx}
\usepackage{setspace}
\setstretch{1.2}

\author{Nicole Loncke \& Lucianne Walkowicz}
\date{\today}
\title{Efficiently Searching for Stellar Flares in Kepler Mission Data}

\begin{document}

\maketitle{}

% Section 1:
% A few corrections: Kepler stares at one one part of the sky, and does
% so for *years* at a time, broken into 3 month chunks. You can refresh
% your Kepler knowledge at kepler.nasa.gov (and you should probably
% include a footnote with that URL as a reference). 
\section{Introduction}
\label{sec:intro}

The Kepler Mission is designed to survey our galaxy in the hopes of
discovering planets in or near the habitable zone of their stars and
determine how many of the billions of stars in our galaxy have such
planets. It essentially stares at a relatively small portion of the
sky for years at a time to gather brightness data about these stars.
In addition to planet detection, this data can be used to gather
properties about the stars themselves.  In this paper we are concerned
with the flaring behaviors of these nearby stars and techniques we can
use to better determine their effect on any orbiting planets.

% \section{Using the Tools}
% \label{sec:tools}
% In order to gather data about to flares, it is necessary to first
% correctly identify the stellar flares within the Kepler data.  These
% events have a very distinct shape and thus can be picked out from the
% light curve data by a program with rudimentary accuracy.  We have
% written such a program in IDL and must manually verify all of the
% events it flags as flares.  To do so, we have also written a Python
% module, \verb|lightcurves|, to aid in the vetting process.

% % Section 2.1 last sentence: this still reads like documentation
% % (because that's where it came from!). Try rephrasing this sentence,
% % perhaps as ``The module functions use the light curve data and flags
% % as inputs for vetting'' or something like that.

% \subsection{Formatting}
% \label{sec:format}

% The \verb|lightcurves| module makes some assumptions about the format
% of the input data.  The original light curve data should be in a file
% containing a whitespace-separated table with time in the first column
% and flux in the second.
% \begin{table}[h]
%   \centering
%   \begin{tabular}{>{\itshape}p{0.2\linewidth} >{\itshape}l}
%        808.51470   &   6338.22 \\
%        808.53514   &   6340.73 \\
%        808.55557   &   6346.89 \\
%        808.57601   &   6341.10 \\
%        808.59644   &   6340.22 \\
%   \end{tabular}
%   \caption{Light curve data sampled from Kepler ID 10068383.}
% \end{table}

% The other fundamental input file is the one containing the potential
% flare events, or ``flags.''  This file contains one column listing
% indices into the time array at the points that mark a suspected event.

% Those two files --- the light curve data and the flags --- are all you
% need to start using these tools.

% % Section 2.2: I'm seeing what you mean by tense changes here-- the ``if
% % you X'' and ``if your aim is X'' stuff is what makes it still read
% % like documentation! Try taking out sentences like this and replacing
% % them with simple statements of what the function actually *does*, such
% % as ``The ltcurve() function reads a single data file and displays it
% % in a plot for inspection''. 


% \subsection{Plotting}
% \label{sec:basic}
% If your aim is simply to generate arrays of the light curve data ---
% one array for time, another for mean-normalized flux --- then use
% \verb|ltcurve()|.  This function takes as its primary argument a
% string of the name of the file containing the Kepler data and returns
% the time and brightness arrays.  By default it also displays the light
% curve corresponding to the file on a time vs. brightness plot, but
% this feature can be switched off by passing the function an optional
% argument.

% If you have \emph{multiple} light curve files and would like to view
% them one at a time, then \verb|ltcurves()| is more appropriate.  Its
% only required argument is a list or array of filename strings.  Note
% that this function does not return any of the data.  In addition, if
% for each of the Kepler data files you have a set of corresponding
% event flags\footnote{You can generate these flags using
%   \texttt{getflags()} and passing it a list of the names of the files
%   holding the flare flags.}, you may use the \verb|flags| keyword
% argument to overplot the potential flares.

% \subsection{Vetting}
% \label{sec:vet}

% Instead of cycling through the light curves with overplotted flags,
% you may find it helpful to inspect and record whether or not the
% marked events could potentially be stellar flares.  In that case, you
% should use \verb|flareshow()|, which writes user input (either 'y',
% 'n', 'm') to two files for later retrieval.  See examples of the
% display in figures \ref{fig:yes}, \ref{fig:no}, and \ref{fig:maybe}.
% \begin{figure}[h!]
%   \caption{A true stellar flare, displayed with flareshow().}
%   \label{fig:yes}
%   \centering
%     \includegraphics[width=\textwidth]{plot_yes_zoom}
% \end{figure}

% \begin{figure}[h!]
%   \caption{A falsely flagged event.}
%   \label{fig:no}
%   \centering
%     \includegraphics[width=\textwidth]{plot_no_zoom}
% \end{figure}

% \begin{figure}[h!]
%   \caption{An indeterminate event.}
%   \label{fig:maybe}
%   \centering
%     \includegraphics[width=\textwidth]{plot_maybe_zoom}
% \end{figure}

% One file contains a space-separated table of the Kepler IDs and the
% corresponding user responses to its events, see Table
% \ref{tab:output}.  The other file contains information about the
% length of each event, as displayed in Table \ref{tab:outputindices}.
% These two files work in conjunction to gather more information about
% the potential flares.
% \begin{table}[h]
%   \centering
%   \begin{tabular}{l l}
%     8848271 &  n \\
%     8908102 &  n \\
%     8953257 &  n  n  n  n  n  n  n  n \\
%     9002237 &  n  n  n  y \\
%   \end{tabular}
% \caption{Example output.txt file.}
% \label{tab:output}
% \end{table}

% \begin{table}[!h]
%   \centering
%   \begin{tabular}{l l}

% 8848271 &  3735 03 \\
% 8908102 &  1757 03 \\
% 8953257 &  1454  6 1610  7 1890  4 2359  3 2516  4 2829  5 2985  6
% 3265  5 \\
% 9002237 &  3337  4 3547  5 3756  3 3967  4 \\
% \end{tabular}
% \caption{Corresponding example output\_indices.txt file.}
% \label{tab:outputindices}
% \end{table}

% Note that before using \verb|flareshow()|, you must have your flags in
% the proper format, generated by \verb|getflags()|.  This helper
% function outputs a nested list of event indices given a list of the
% names of the files containing the flags.

\section{Initial Approach}
\label{sec:initial}
In order to gather data about the flares, it is necessary to first
correctly identify the stellar flares within the Kepler data.  These
events have a very distinct shape and thus can be picked out from the
light curve data by a simple program with rudimentary accuracy.  We
have written such a program in IDL that takes note of points in the
light curve where the brightness exceeds a certain threshold.  Before
proceeding to collect data on the flares, we must first manually
verify all of the events it flags as flares.

To make this verification process easier, we have also written a
Python module, \verb|lightcurves.py|.  It contains functions that
display the light curves with the flagged points marked, in addition
to accepting and recording user input about whether the flags are
truly flare events.  We employed a combination of these tools to
inspect over 300 events, \{\} of which we determined to be flares.


% Section 4: a bit of clarity for the intro here: the IDL flare
% detection program detects parts of the lightcurve where the star
% becomes brighter suddenly, flagging these events as possible
% flares. It's not really correct to say that it's trained to do
% anything-- our classifiers are trained, the event detections are just
% dumb threshold cuts where the program flags something if it got
% brighter for a while above some threshold. So what our human brains
% are doing is similar to what the classifier is doing, but not really
% what the event detection software is doing. I would try rephrasing and
% saying something along the lines of that when humans vet the flare
% events by eye, the brain is comparing the flare event to a set of
% ideas (a model) of what a real flare should look like in the
% data. Those ideas can be quantified into metrics, and then used as
% inputs for the machine classification.

\section{Machine Learning}
\label{sec:ml}

This method of visually checking each event is slow, however, and the
major bottleneck for data analysis. The IDL flare detection program
that produces the flare flags recognizes some data metrics but does
not correctly identify events with high accuracy.  Our human brains
allow us to do the same thing but with more nuance.  You can see some
examples for yourself in figures 1,2, and 3.  If we could write
another program to recognize the same patterns that humans so easily
detect in the light curves then we could nearly entirely automate the
data-gathering process---from the raw light curve to having
information about flares with high confidence.  To accomplish this
goal, we trained a variety of classifiers on metrics from each
potential flare event and their respective light curves.

% Section 4.1 small rephrase in the 1st sentence: you should probably
% say something like ``The first task was to make quantitative
% measurements of features for each flare'' or something, the data
% itself is ``quantitative'' already. You can use features and metrics
% interchangeably (I think-- I always do, so hopefully that's correct!)
\subsection{Training}
\label{sec:train}
Our first task was to gather quantitative data about the stellar
flares to feed into the classifier.  In total we use 10 metrics.
\begin{enumerate}[(1)]
\item \emph{amplitude}: the range of the entire light curve.  Stars
  with great stellar variability tend to be more magnetically active
  than those without.  We expect high light curve amplitude to
  correlate with real flares.
\item \emph{number of events}: Light curves that have many flagged
  events tend to have real flares, so we expect a high number of
  events to correlate with real flares.
\item \emph{standard deviation}: The standard deviation of the entire
  light curve with stellar variability subtracted.  It may be useful
  to feed the classifier more information about the light curve at
  large.
\item \emph{consecutive points}: Sometimes there are gaps in the
  Kepler data.  Kepler must rotate and point its antenna towards Earth
  to send its light curve data roughly every month.  When the
  satellite begins recording again, there may be a sudden increase in
  brightness that resembles a flare but isn't.  In order to avoid
  marking these as true flares we check whether the time intervals are
  evenly spaced across the event.
\item \emph{kurtosis}: The kurtosis measures the ``peakedness'' of a
  flare event.  A sharp increase and decrease in brightness is likely
  to indicate a true flare, though the decay ought to be more gradual
  than the incline.
\item \emph{midpoint check}: A stellar flare typically requires a
  monotonic increase then monotonic decrease in brightness.  Ensuring
  that the middle point is higher than the beginning and end points of
  the event is one way to rule out falsely marked events.
\item \emph{second derivative}: Smoothing over the flagged event, is
  the light curve locally concave up or down?  The second derivative
  of the window around the potential flare can capture the shape of a
  light curve in the neighborhood of an event.
\item \emph{skew}: Skewness is a measure of the asymmetry of the event
  brightness---is the flare left-leaning or right-leaning?  Because
  flares are characterized by very quick increases in brightness
  followed by a slow decay, left-leaning events (and therefore those
  with negative skew) are more likely to be true flares.
\item \emph{slope}: Is the brightness of the star generally increasing
  or decreasing at the time of the event?  This metric measures the
  slope of the line formed by connecting the point at the beginning of
  the flare window to the point at the end of the flare window.
  time of the event?
\item \emph{slope ratio}: We also compute the ratio of the light
  curve's slope just before the event begins and the slope just after
  it ends.  We hope to capture more information about the local shape
  of the light curve with this metric.
\end{enumerate}
These data were gathered for 315 potential flaring events that we had
previously labelled by-eye.  Using these samples we formed a training
set of 150 events and a test set of 165 events.

\subsection{Classification Performance}
\label{sec:class}
We use Python's scikit-learn package for our machine learning
framework.  While it comes equipped with a suite of regression,
clustering, and dimensionality-reduction tools, we are only concerned
with classification.  Our target classes are 'y' for definitely a
flare, 'n' for not a flare, and 'm' for any indeterminate events.

To quantitatively compare the classifiers, it is important that we
define a few statistics related to their performance.  We say the
\emph{precision}, or \emph{efficiency}, is the fraction of events
classified as a given type ('y', 'n', 'm') that are truly of that
type.  The \emph{recall}, or \emph{completeness} of the classifier is
the fraction of objects that are truly of a given type that it
classifies as that type.  The $F_1$ score is a weighted average of
recall and precision.

Though we ideally seek high scores for both precision and recall, for
our purposes precision is the more important metric.  Because we have
many events in our dataset it is better to correctly identify a small
number of flares than to find many true flares at the expense of false
positives.


\subsubsection{Support Vector Classification}
\label{sec:svc}
For our first attempt we used support vector classification as
packaged in \verb|sklearn.svm.SVC|. We initially used a linear kernel
SVM. This is a simple classification method which assumes that there
is a hyperplane that separates the data in the feature space, which is
10-dimensional in our case.  We trained our linear SVC on 150 flare
events and then used it to predict the status of 165 events.  This
classifier had a precision of 63\% for classifying true flare events.
While the results were better than a coin toss, we sought to improve
the classifier performance.
\begin{table}
  \centering
  \begin{tabular}[!htbp]{c|c c c c}
        & precision &recall &$F_1$-score &support \\ \hline
    n   & 0.62      &0.87   &0.72     &60      \\
    y   & 0.63      &0.69   &0.66     &65      \\
    m   & 0.50      &0.12   &0.20     &40      \\ \hline
    avg & 0.60      &0.62   &0.57     &165     \\
  \end{tabular}
  \caption{Linear kernel performance with the test set.}
  \label{tab:lintest}
\end{table}

Sometimes the data cannot be linearly separated within the given
feature-space.  Alternate SVM kernels implicitly map the data to
higher dimensions to find a separating hyperplane.  To do that, we
incorporated the popular radial basis function (RBF) kernel into our
support vector machine model.  While support vector classification
with an RBF kernel does decently when predicting the flares it has
already seen, it does not outperform when predicting unseen events,
which is ultimately what is important. The results are charted in
tables \ref{tab:rbftrain} and \ref{tab:rbftest}.

\begin{table}
  \centering
  \begin{tabular}[!htbp]{c|c c c c}
        & precision &recall &$F_1$-score &support \\ \hline
    n   & 0.73      &0.81   &0.77     &57      \\
    y   & 0.74      &0.89   &0.81     &61      \\
    m   & 0.71      &0.31   &0.43     &32      \\ \hline
    avg & 0.73      &0.73   &0.71     &150     \\
  \end{tabular}
  \caption{Reconstructing the training set with RBF kernel.}
  \label{tab:rbftrain}

  \begin{tabular}[!htbp]{c|c c c c}
        & precision &recall &$F_1$-score &support \\ \hline
    n   & 0.61      &0.88   &0.72     &60      \\
    y   & 0.62      &0.60   &0.61     &65      \\
    m   & 0.33      &0.12   &0.18     &40      \\ \hline
    avg & 0.55      &0.59   &0.55     &165     \\
  \end{tabular}
  \caption{RBF kernel performance on the testing set.}
  \label{tab:rbftest}
\end{table}


\subsubsection{Random Forest Classifier}
\label{sec:randfor}
We performed the same task using random forest classification, as
packaged in \verb|sklearn.ensemble|.  The algorithm constructs many
decision trees based on the inputs it sees during the training phase,
then uses those trees to predict the categories of unseen
samples. While it performed superbly on the training set, it performed
no better than our support vector machines on the test set.

\begin{table}
  \centering
  \begin{tabular}[!htbp]{c|c c c c}
       & precision &recall &$F_1$-score &support \\ \hline
    n  & 1.00      &0.92   &0.99     &57      \\
    y  & 0.95      &1.00   &0.98     &61      \\
    m  & 1.00      &0.94   &0.97     &32      \\ \hline
    avg& 0.98      &0.98   &0.98     &150     \\
  \end{tabular}
  \caption{Reconstructing the training set with random forest classification method.}

  \begin{tabular}[!htbp]{c|c c c c}
       & precision &recall &$F_1$-score &support \\ \hline
    n  & 0.63      &0.77   &0.69     &60      \\
    y  & 0.60      &0.72   &0.66     &65      \\
    m  & 0.29      &0.10   &0.15     &40      \\ \hline
    avg& 0.54      &0.59   &0.55     &165     \\
  \end{tabular}
  \caption{Random forest method performance on the testing set.}
\end{table}

\subsubsection{Linear Discriminant Analysis}
\label{sec:lda}
Lastly, we used the linear discriminant analysis method (LDA), which
attempts to model the difference between the classes as linear
combinations of the features.  LDA is closely related to principal
component analysis in that it can also be used for dimensionality
reduction, but we employed LDA only for its capabilities as a linear
classifier.
\begin{table}
  \centering
  \begin{tabular}[!htbp]{c|c c c c}
       & precision &recall &$F_1$-score &support \\ \hline
    n  & 0.72      &0.86   &0.78     &57      \\
    y  & 0.70      &0.84   &0.76     &61      \\
    m  & 0.67      &0.19   &0.29     &32      \\ \hline
    avg& 0.70      &0.71   &0.67     &150     \\
  \end{tabular}
  \caption{Reconstructing the training set with LDA.}

  \begin{tabular}[!htbp]{c|c c c c}
        & precision &recall &$F_1$-score &support \\ \hline
    n   & 0.63      &0.83   &0.72     &57      \\
    y   & 0.60      &0.71   &0.65     &61      \\
    m   & 0.56      &0.12   &0.20     &32      \\ \hline
    avg & 0.60      &0.61   &0.57     &150     \\
  \end{tabular}
  \caption{LDA performance on the testing set.}
\end{table}
Our LDA classifier performed with 60\% precision and 71\% completeness
for the true flares, which is comparable to the other three methods.

% Conclusions: You should note here that while the human element hasn't
% been completely removed, the amount of human intervention required is
% MUCH, much less using this method than it would be. Remember that
% you've only had to look at 300ish flares-- but there are roughly
% 30,000 stars we can apply this to for every 3 month chunk of Kepler
% data over the past five years. That would be impossible to do if
% humans had to do all the vetting by eye!
\section{Conclusion}
\label{sec:conc}
We set out to gather data about flaring stars in order to explore the
impact that these stars might have on the conditions of nearby
planets.  Processing the light curves to identify flares, however,
required a large human component in the form of looking at each curve
and manually recording the status of each event.  In an effort to
automate flare detection, we employed machine learning techniques to
classify a set of pre-vetted events.  Of the four classifiers
used---SVM with linear kernel, SVM with radial basis function kernel,
random forest, and linear discriminant analysis---performance metrics
suggest that the simple linear support vector machine is the best
choice for our task, as it has the best precision for the test data.

There is still much room for performance improvement, however.  As
with many classification problems, increasing the size of the training
set can significantly improve prediction accuracy.  Doing so requires
more labelled data which means we must, ironically, vet more potential
flares by eye.  As a diagnostic for expected improvement we can plot a
learning curve (accuracy as a function of training set size) for each
of the different classifiers.  It may be that the classifiers have
simply not seen enough variety in the training set to be able to
classify new inputs. Alternatively, if the learning curves have
already plateaued with a training set of 150 samples then providing
more samples won't significantly improve prediction accuracy.

The advantage of being able to test the classifiers on labelled data
is that we can see how many of the flares are being incorrectly
identified.  With this knowledge, there may be a way to correct for
those flares that are misjudged by our algorithms.

Currently, we have yet to completely remove the human element from
detecting light curves, but with the the machine learning techniques
mentioned we have higher hopes for doing so.
% Nice work thus far-- it would be cool to include those learning curves
% if you have them! I realize you have finals coming up but if you get a
% chance I'd really like to see how they look (and I don't think it
% would take too long). 
% Cheers
% L

\end{document}
\documentclass{article}
% \usepackage{showframe}
\usepackage{caption}
\usepackage{array}
\author{Nicole Loncke}
\date{\today}
\title{Workflow for Stellar Flare Lightcurve Analysis}

\begin{document}

\maketitle{}

\section{Formatting}
\label{sec:format}

The lightcurves module makes some assumptions about the format of the
input data.  The original light curve data should be in a file
containing a whitespace-separated table with time in the first column
and flux in the second.
\begin{table}[h]
  \centering
  \begin{tabular}{>{\itshape}p{0.2\linewidth} >{\itshape}l}
       808.51470   &   6338.22 \\
       808.53514   &   6340.73 \\
       808.55557   &   6346.89 \\
       808.57601   &   6341.10 \\
       808.59644   &   6340.22 \\
       808.61688   &   6340.61 \\
       808.63731   &   6342.13 \\
       808.65774   &   6349.23 \\
       808.67818   &   6343.68 \\
       808.69861   &   6334.51 \\
       808.71905   &   6337.67 \\
       808.73948   &   6348.09 \\
  \end{tabular}
  \caption{Actual light curve data sampled from kid10068383.txt.}
\end{table}

The other fundamental input file is the one containing the potential
flare events, or ``flags.''  This file simply contains a column
listing indices into the time array at the points that mark a
suspected event.

\begin{table}[h]
  \centering
  \begin{tabular}{>{\itshape}p{0.2\linewidth}l}
         350 \\
         351 \\
         352 \\
         370 \\
         371 \\
         372 \\
         373 \\
         374 \\
         375 \\
         549 \\ 
         550 \\ 
         551 \\
         552 \\
         553 \\
\end{tabular}
\caption{Actual flags sampled from the file corresponding to
  kid10068383.}
\end{table}

Those two files --- the light curve data and the flags --- are all you
need to start using these tools.


\section{Plotting}
\label{sec:basic}

If your aim is simply to generate numpy arrays of the lightcurve data
--- one array for time, another for mean-normalized flux --- then use
\verb|ltcurve()|.  This function takes as its primary argument a
string of the name of the file containing the Kepler data and returns
the time and brightness arrays.  By default it also displays the light
curve corresponding to the file on a time vs. brightness plot, but
this feature can be switched off by passing the function an optional
argument.

If you have \emph{multiple} light curve files and would like to view
them one at a time, then \verb|ltcurves()| is more appropriate.  Its
only required argument is a list or array of filename strings.  Note
that this function does not return any of the data.  In addition, if
for each of the Kepler data files you have a set of corresponding
event flags\footnote{You can generate these flags using
  \texttt{getflags()} and passing it a list of the names of the files
  holding the flare flags.}, you may use the \verb|flags| kwarg to
overplot the potential flares.


\section{Vetting}
\label{sec:vet}

Instead of cycling through the light curves with overplotted flags,
you may find it helpful to inspect and record whether or not the
marked events could potentially be stellar flares.  In that case, you
should use \verb|flareshow()|, which writes user input (either 'y',
'n', 'm') to two files for later retrieval.  One file contains a
space-separated table of the Kepler IDs and the corresponding user
responses to its events.  The other file contains information about
the length of each event.  These two files work in conjunction to
gather more information about the potential flares.

\begin{table}[h]
  \centering
  \begin{tabular}{l l}

8848271 &  n \\
8908102 &  n \\
8953257 &  n  n  n  n  n  n  n  n \\
9002237 &  n  n  n  y \\
\end{tabular}
\caption{Example output.txt file.}
\end{table}



\begin{table}[h]
  \centering
  \begin{tabular}{l l}

8848271 &  3735 03 \\
8908102 &  1757 03 \\
8953257 &  1454  6 1610  7 1890  4 2359  3 2516  4 2829  5 2985  6
3265  5 \\
9002237 &  3337  4 3547  5 3756  3 3967  4 \\
\end{tabular}
\caption{Corresponding example output\_indices.txt file.}
\end{table}

Note that before using \verb|flareshow()|, you must have your flags in
the proper format, generated by \verb|getflags()|.  This helper
function outputs a nested list of event indices given a list of the
names of the files containing the flags.

\section{Data Processing}
\label{sec:advanced}

After evaluating the marked events by eye, quantifying data about the
remaining candidates is the next step.  Assuming that you used
\verb|flareshow()| for vetting, you now have two output files for the
set of lightcurve data.  Use the helper function \verb|getEvents()|,
which reads from these output files to pare down your list of flags to
only those that have been marked with 'y' or 'm' depending on how you
set the kwargs.

To calculate the cumulative brightness of each event found within a
single light curve, use \verb|intFlare()|.  This function returns an
array of the integrated brightness over the course of the events, an
array of the duration of the events (in hours), and the peak
brightnesses of each event.  It is important to note that this
function assumes you have vetted the flags already and are only
providing those about which you would like to find more information
(ie, you have used \verb|getEvents()|).

\end{document}